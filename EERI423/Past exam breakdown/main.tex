\documentclass{article}
\usepackage{multicol} 
\usepackage{amsmath} 
\author{MJ Bezuidenhout}
\title{EERI423 Exam breakdown 2013-2016}

\begin{document}
\begin{multicols}{2}
\maketitle
\section{2016}
\begin{itemize}
	\item Question 1
		\subitem Elements of a communication system 
		\subitem Block diagram of a phase locked loop synth
	\item Question 2
		\subitem Square law mixer
		\subitem Receiver noise calcs
		\subitem Image frequency
		\subitem Block diagram modern direct conversion Receiver
	\item Question 3
		\subitem Duplexing
		\subitem Shannon-Hartley eqn
		\subitem Fading margin
		\subitem LTE
	\item Question 4
		\subitem Power budget
	\item Question 5
		\subitem Cellular systems
	
\end{itemize}
\section{2015}
\begin{itemize}
	\item Question 1
		\subitem Baseband vs broadband
		\subitem multiplexing
		\subitem Quantizing noise
	\item Question 2
		\subitem Block diagram of a modern digital transmitter
		\subitem Block diagram of a direct digital synthesizer
		\subitem Receiver noise calcs
		\subitem theory
			\subsubitem SDR	
			\subsubitem Mixer in a digital synth	
			\subsubitem Automatic gain control	
	\item Question 5
		\subitem Cellular systems
	
\end{itemize}

\section{2014}
\begin{itemize}
	\item Question 1
		\subitem Form factor of a band pass filter
		\subitem Insertion Loss
		\subitem Multiplexing
		\subitem Adding up gains
	\item Question 2
		\subitem Block diagram of a typical FM transmitter
		\subitem Variable modulus transmitter
		\subitem Block diagram of a double heterodyne transmitter
		\subitem Receiver noise calcs
	\item Question 3
		\subitem theory
			\subsubitem RZ encoding
			\subsubitem Multiplexing, (``normal binary channel?!'')
			\subsubitem GMSK
			\subsubitem FDD vs TDD
			\subsubitem OFDM
		\subitem Shannon-Hartley eqn
	\item Question 4
		\subitem Path loss
	\item Question 5
		\subitem Cellular systems
	
\end{itemize}
\section{2013}
\begin{itemize}
	\item Question 1
		\subitem 4 elements of any communication channel
		\subitem Compander w.r.t speech signals
		\subitem Compander calc
	\item Question 2
		\subitem Block diagram of a modern digital transmitter
		\subitem Problem with fixed prescalers in synths
		\subitem Receiver noise calcs
	\item Question 3
		\subitem 2 types of spread spectrum
		\subitem 3 advantages of spread spectrum
		\subitem Block diagram of a carrier recovery circuit for BPSK modulation
		\subitem Shannon-Hartley eqn
	\item Question 4
		\subitem Path loss
	\item Question 5
		\subitem Cellular systems
	
\end{itemize}
\section{List of block Diagrams}

\section{Cellular systems}

\section{Path Loss}

\section{Receiver Noise}

\section{Shannon Hartley}
\begin{itemize}
	\item $C=2B$ Where C is the capacity in bps and B is the bandwidth in Hertz
	\item It assumes only 2 encoding levels are used, i.e. High=1 and low=0
	\item $C=2Blog_2N$ Where N is the number of encoding levels
	\item $C=Blog_2\left(1+\frac{S}{N}\right)$Where $\frac{S}{N} $ is the SNR \emph{As a ratio, not as DB}
\end{itemize}
\end{multicols}
Consider compiling exam theory questions into flash cards, current flash cards too many
\end{document}
