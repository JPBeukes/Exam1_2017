\documentclass{article}

\usepackage{booktabs}
\usepackage{tabularx}
\usepackage{ltablex} 
\usepackage{longtable}
\usepackage{graphicx}
\usepackage{seqsplit}% http://ctan.org/pkg/seqsplit
\usepackage{array}
\usepackage{enumitem}
\usepackage{amsmath}

\usepackage{geometry}
 \geometry{
 a4paper,
 total={170mm,257mm},
 left=15mm,
 top=20mm,
 }
 
\newcounter{count}
\setcounter{count}{1}
\setlist{nosep}

\newcommand{\qa}[2]{\thecount){#1} & \thecount) {#2} \\ \stepcounter{count}}

\newcommand{\newheading}[1]{\end{longtable} 
\textbf{#1}
\setcounter{count}{1}%comment out not to reset counter at each new heading
\begin{longtable}{m{250pt} m{250pt}}}



\begin{document}





 
{\huge REII414 Semester test 1 flash cards}
\large %comment out to make text smaller
\setlength{\tabcolsep}{0.8em}
\setlength{\extrarowheight}{0.5em}
\begin{longtable}{m{250pt} m{250pt}}


\textbf{SU1}\\
%Database terminology
\qa{Define a Primary Key}{An attribute that \emph{uniquely} defines each row}
\qa{Define a foreign Key}{An attribute whose value matches the primary key of a related table}
\qa{Define entity integrity}{The property of a relational table that
guarantees that each entity has a unique value in a primary
key and that there are no null values in the primary key.}
\qa{Define referential integirty}{A condition by which a dependent
table�s foreign key must have either a null entry or a
matching entry in the related table. Even though an attribute
may not have a corresponding attribute, it is impossible to
have an invalid entry.}
\qa{Define an ERD}{Refers to the entity relationship
diagram resulting from the application of extended entity
relationship concepts that provide additional semantic
content zrein the ER model.}
\qa{List the types, and uses for the different crow's feet in the abovementioned diagram}{
\begin{itemize}
	\item Zero or many  
	\item One or many
	\item ONe and only one
	\item Zero or one
	]i
\end{itemize}
}
\qa{Define a derived attribute}{ an attribute whose value is
calculated (derived) from other attribute} 
%data vs information
\qa{Briefly differentiate between data and information}{Data are raw facts, information is the result of processing data to reveal its meaning}
%development of database systems
\qa{State the 3 ways in which a business might manage its data}{
\begin{itemize}
	\item A Manual (paper) file system
	\item A computerized file system
	\item A database system
\end{itemize}
}
\qa{State 3 problems with computerized file systems}{
\begin{itemize}
	\item Structural and data dependence
	\item Data redundancy
	\item Lack of design and data-modeling skills
	
\end{itemize}
}
%Role of the DBMS
\qa{list 9 roles of the DBMS}{
\begin{itemize}
	\item Data dictionary management
	\item Data Storage management
	\item Data transformation and presentation
	\item Security Management	
	\item Multiuser access control
	\item Backup and recovery management
	\item Data integrity management
	\item Database access languages and application programming interface
	\item Database communication interfaces
\end{itemize}
}
%structure of the file based system and the entity relationship system
\qa{define structural dependence}{Access to a file is dependent on its structure, } 
\qa{define structural independence}{When access to a file is not affected by changes in its structure} 
\qa{define data dependence}{When access to a file is dependent on its data, for example an appication that relies on the data-type in a file to be an integer.} 
\qa{define data independence}{When it is possible to change the data in a file without affecting access by an application} 
\qa{describe the difference between the logical and physical data format}{The logical data format is the format in which a human reads the data, whereas the physical data format defines the way a computer works with the data} 
%advantages and disadvantages of different systems
%Database system environment
\qa{state 5 components of the database system environment}{
\begin{itemize}
	\item Hardware
	\item Software
	\item People
	\item Procedures
	\item Data
\end{itemize}
} 

\newheading{SU2}
%entities and features
\qa{Define an entity}{something to which data will be attributed, usually corresponds to a table} 
\qa{Define a feature}{An attribute of an entity} 
%features of a relational table
\qa{list 8 Characteristics of a relational table}{
\begin{itemize}
	\item A table is perceived as a two-dimensional structure composed of rows and columns.
	\item Each table row (tuple) represents a single entity occurrence within the entity set.
	\item Each table column represents an attribute, and each column has a distinct name.
	\item  Each row/column intersection represents a single data value.
	\item All values in a column must conform to the same data format
	\item Each column has a specific range of values known as the attribute domain.
	\item The order of the rows and columns is immaterial to the DBMS. 
	\item Each table must have an attribute or a combination of attributes that uniquely identifies each row. 
\end{itemize}
} 
%data types!!!!NOT DONE!!!!
%define the concept of keys
\qa{define the concept of a key}{An entity identifier based on the concept of functional
dependence} 
%identify and use different types of keys
\qa{define a  Superkey}{an
attribute (or combination of attributes) that uniquely
identifies each entity in a table}
\qa{define a Primary key}{a
candidate key selected as a unique entity identifier}
\qa{define a Secondary key}{a key that is used strictly for data retrieval
purposes. For example, a customer is not likely to know his
or her customer number (primary key), but the combination
of last name, first name, middle initial, and telephone
number is likely to make a match to the appropriate table
row}
\qa{define a Foreign key}{an attribute (or combination of attributes)
in one table whose values must match the primary key in
another table or whose values must be null.}
\qa{define a Candidate key}{a minimal
superkey, that is, one that does not contain a subset of
attributes that is itself a superkey}
\qa{list 6 desirable attributes of a primary key}{
\begin{itemize}
	\item Stable : does not change over time
	\item Minimal: Fewest attributes necessary
	\item Factless: no hidden information	
	\item Definitive: Value always exists
	\item Accessible: Available when data created
	\item Unique: Absolutely no duplicates
\end{itemize}
} 
 
%integrity rules
\qa{briefly define Data integrity}{In a relational database, refers to a condition
in which the data in the database is in compliance with all
entity and referential integrity constraints.} 
\qa{Define entity integrity}{The property of a relational table that
guarantees that each entity has a unique value in a primary
key and that there are no null values in the primary key.}
\qa{Define referential integrity}{A condition by which a dependent
table�s foreign key must have either a null entry or a
matching entry in the related table. }

%database computations !!!NOT DONE!!!
%the database dictionary
\qa{Briefly describe the data dictionary}{ provides a detailed description of all tables found within the user/designer-created database.IN other words, the database dictionary contains metadata} 
\qa{what would a minimal example of a data dictionary contain}{ at least all of the attribute names and characteristics for each table in the system} 
%entity relationship diagrams
%redundancy




\newheading{SU3}
\qa{Define a Functional dependency}{Attribute A determines attribute B (that is, B is functionally dependent on A) if all
of the rows in the table that agree in value for attribute A also agree in value for
attribute B} 
\qa{Define a Partial dependency}{A dependency that exists when the determinant is only part of the primary key [if (A,B) $\to$ (C,D) and B $\to$ C Where (A,B) is the PK]} 
\qa{Define a transitive dependency}{Dependencies such that X $\to$ Y and Y $\to$ Z form a transitive dependency.In general, transitive dependencies are dependencies between non-primary attributes} 
\qa{list the 5 normal forms}{
\begin{itemize}
	\item First Normal Form (1NF)
	\item Second Normal Form (2NF)
	\item Third Normal Form (3NF)
	\item Boyce-Codd normal form(BCNF)
	\item Fourth Normal Form (4NF)
	]i
\end{itemize}
} 
\qa{give the requirements for 1NF}{Table format, no repeating groups, and PK identified} 
\qa{give the requirements for 2NF}{1NF and no partial dependencies}  
\qa{give the requirements for 3NF}{2NF and no transitive dependencies} 
\qa{give the requirements for BCNF}{Every determinant is a candidate key (special case of 3NF) } 
\qa{give the requirements for 4NF}{3NF and no independent multivalued dependencies} 
%describe basic modelling concepts;
%describe degrees of data abstraction for the different database models;
%define and describe the conceptual, internal, external and physical abstraction of data;
%determine relations and apply these between entities;
%describe and determine connectivity and cardinality between related entities;
%interpret and award E-R symbols;
%apply entity classification as a super type or sub type;
%know and apply external key rules;
%apply appropriate modelling methods such as Chen and Crow's foot;
%know database modelling conventions;
%apply design compromises based on sound principles and conventions.
\newheading{SQL queries}
\qa{Briefly describe select syntax}{SELECT $<$fields$>$ FROM$<$tables$>$ WHERE$<$conditions$>$ } 
\qa{Briefly describe update syntax}{UPDATE$<$tabls$>$ SET $<$field = new$>$ WHERE $<$condition$>$  }
\qa{state the SQL function for returning the average}{AVG()} 
\qa{state the SQL function for returning the number of rows}{COUNT()} 
\qa{state the SQL function for returning the first value}{FIRST()} 
\qa{state the SQL function for returning the last value}{LAST()} 
\qa{state the SQL function for returning the largest value}{MAX()} 
\qa{state the SQL function for returning the Smallest value}{MIN()} 
\qa{state the SQL function for returning the sum}{SUM()} 
\qa{state the SQL function for returning a numeric field to specific number of decimals }{ROUND()} 

\end{longtable}

\end{document}
