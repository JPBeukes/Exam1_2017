\documentclass{article}
\usepackage{hyperref} 
\title{REII414 past papers comaprison\\ \small
For the latest version of this document, visit https://github.com/p0te/Exam1\_2017/REII414}
\author{MJ Bezuidenhout}
\begin{document}
\maketitle
\section{2015}
\begin{itemize}
	\item 6 questions:
		\subitem 1:sql Theory
		\subitem 2:entity and referential integrity
		\subitem 3: ERD diagram
		\subitem 4: Dependecies, noprmailzation
		\subitem 5: difficult qrys (joins, sums etc)
		\subitem 6: essay: final year project
\end{itemize}
\section{2016}
\begin{itemize}
	\item 5 questions:
		\subitem 1: sql Theory 
		\subitem 2: entity and referential integrity
		\subitem 3: Dependencies, normalization
		\subitem 4: Difficult queries,(joins, sums etc)
		\subitem 5: essay: game
\end{itemize}
\section{Sql Theory(See also, flash cards)}
\begin{itemize}
	\item Data vs information
	\item Right vs left joins
	\item ERD principles(Arrow types, redundant relationship)
	\item Derived attribute
	\item attributes of a PK
	\item triggers
	\item SQL injection
\end{itemize}
\section{Entity and referential integrity}
\begin{description}
	\item[Data Integrity] In a relational database, refers to a condition
in which the data in the database is in compliance with all entity and referential integrity constraints 
	\item[Entity Integrity]  The property of a relational table that guarantees that each entity has a unique value in a
primary key and that there are no null values in
the primary key.
	\item[Referential Integrity] A condition by which a dependent tables foreign key must have either a null entry or a matching entry in the related table
\end{description}
\section{Dependencies, Normalization}
\subsection{Dependencies}
\begin{description}
	\item[Functional Dependency]  Attribute A determines attribute B (that is, B
is functionally dependent on A) if all of the rows
in the table that agree in value for attribute A
also agree in value for attribute B
	\item[Partial Dependency] A dependency that exists when the determinant is only part of the primary key [if (A,B) $\rightarrow$
(C,D) and B $\rightarrow$ C Where (A,B) is the PK]
	\item[Transitive Dependency] Dependencies such that X $\rightarrow$ Y and Y $\rightarrow$ Z
form a transitive dependency.In general, transitive dependencies are dependencies between nonprimary attributes
\end{description}


\subsection{Normalization}
\begin{description}
	\item[1NF]  Table format, no repeating groups, and PK
identified
	\item[2NF]  1NF and no partial dependencies
	\item[3NF]  2NF and no transitive dependencies
	\item[BCNF] Every determinant is a candidate key (special
case of 3NF)
	\item[4NF]3NF and no independent multivalued depen-
dencies 
\end{description}
\section{Difficult queries(some shown, NOT ALL)}
\begin{description}
	\item[SELECT]SELECT $<$fields$>$ FROM$<$tables$>$ WHERE<conditions> 
	\item[UPDATE] UPDATE$<$table$>$ SET $<$field = new$>$WHERE $<$condition$>$
	\item[CREATE] CREATE $<$table\_name$>$($<$col1$>$ $<$datatype$>$ , $<$col2$>$ $<$datatype$>$)

\end{description}
\section{Essay question}
\begin{itemize}
	\item Database Design
		\subitem ERD
	\item Web programming
		\subitem UI
		\subitem data-interfacing		
		\subitem security
\end{itemize}
\end{document}
